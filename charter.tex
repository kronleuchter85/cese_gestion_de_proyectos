\documentclass[
11pt, % The default document font size, options: 10pt, 11pt, 12pt
codirector, % Uncomment to add a codirector to the title page
]{charter} 




% El títulos de la memoria, se usa en la carátula y se puede usar el cualquier lugar del documento con el comando \ttitle
\titulo{Robot de exploración ambiental} 

% Nombre del posgrado, se usa en la carátula y se puede usar el cualquier lugar del documento con el comando \degreename
\posgrado{Carrera de Especialización en Sistemas Embebidos} 
%\posgrado{Carrera de Especialización en Internet de las Cosas} 
%\posgrado{Carrera de Especialización en Intelegencia Artificial}
%\posgrado{Maestría en Sistemas Embebidos} 
%\posgrado{Maestría en Internet de las cosas}

% Tu nombre, se puede usar el cualquier lugar del documento con el comando \authorname
\autor{Ing. Gonzalo F. Carreño} 

% El nombre del director y co-director, se puede usar el cualquier lugar del documento con el comando \supname y \cosupname y \pertesupname y \pertecosupname
\director{Esp. Ing. Sergio Alberino}
\pertenenciaDirector{FIUBA} 
% FIXME:NO IMPLEMENTADO EL CODIRECTOR ni su pertenencia
\codirector{CODIRECTOR} % para que aparezca en la portada se debe descomentar la opción codirector en el documentclass
\pertenenciaCoDirector{FIUBA}

% Nombre del cliente, quien va a aprobar los resultados del proyecto, se puede usar con el comando \clientename y \empclientename
\cliente{Lic. Mariano Landini}
\empresaCliente{-}

% Nombre y pertenencia de los jurados, se pueden usar el cualquier lugar del documento con el comando \jurunoname, \jurdosname y \jurtresname y \perteunoname, \pertedosname y \pertetresname.
\juradoUno{Nombre y Apellido (1)}
\pertenenciaJurUno{pertenencia (1)} 
\juradoDos{Nombre y Apellido (2)}
\pertenenciaJurDos{pertenencia (2)}
\juradoTres{Nombre y Apellido (3)}
\pertenenciaJurTres{pertenencia (3)}
 
\fechaINICIO{13 de marzo de 2023}		%Fecha de inicio de la cursada de GdP \fechaInicioName
\fechaFINALPlan{18 de mayo de 2023} 	%Fecha de final de cursada de GdP
\fechaFINALTrabajo{20 de noviembre de 2023}	%Fecha de defensa pública del trabajo final


\begin{document}

\maketitle
\thispagestyle{empty}
\pagebreak


\thispagestyle{empty}
{\setlength{\parskip}{0pt}
\tableofcontents{}
}
\pagebreak


\section*{Registros de cambios}
\label{sec:registro}


\begin{table}[ht]
\label{tab:registro}
\centering
\begin{tabularx}{\linewidth}{@{}|c|X|c|@{}}
\hline
\rowcolor[HTML]{C0C0C0} 
Revisión & \multicolumn{1}{c|}{\cellcolor[HTML]{C0C0C0}Detalles de los cambios realizados} & Fecha      \\ \hline
0      & Creación del documento.                                 &\fechaInicioName \\ \hline
1      & Se completa hasta el punto 5 inclusive.                 & 14 de marzo de 2023 \\ \hline
2      & Se completa hasta el punto 9 inclusive	y se agregan correcciones.				& 21 de marzo de 2023 \\ \hline
3      & Se completa hasta el punto 12 inclusive y se agregan correcciones. 				& 28 de marzo de 2023 \\ \hline
4      & Se completa hasta el punto 15 inclusive y se agregan correcciones. 				& 4 de abril de 2023 \\ \hline
%		  Se puede agregar algo más \newline
%		  En distintas líneas \newline
%		  Así                                                    & dd/mm/aaaa \\ \hline
%3      & Se completa hasta el punto 11 inclusive                & dd/mm/aaaa \\ \hline
%4      & Se completa el plan	                                 & dd/mm/aaaa \\ \hline
\end{tabularx}
\end{table}

\pagebreak



\section*{Acta de constitución del proyecto}
\label{sec:acta}

\begin{flushright}
Buenos Aires, \fechaInicioName
\end{flushright}

\vspace{2cm}

Por medio de la presente se acuerda con el \authorname\hspace{1px} que su Trabajo Final de la \degreename\hspace{1px} se titulará ``\ttitle'', consistirá esencialmente en \textcolor{black}{la implementación de un prototipo de un sistema embebido para un dispositivo móvil controlado a distancia con funcionalidades que permiten explorar el entorno}, y tendrá un presupuesto preliminar estimado de \textcolor{black}{760} h de trabajo y \textcolor{black}{\$19.240}, con fecha de inicio \fechaInicioName\hspace{1px} y fecha de presentación pública \fechaFinalName.

Se adjunta a esta acta la planificación inicial.

\vfill

% Esta parte se construye sola con la información que hayan cargado en el preámbulo del documento y no debe modificarla
\begin{table}[ht]
\centering
\begin{tabular}{ccc}
\begin{tabular}[c]{@{}c@{}}Dr. Ing. Ariel Lutenberg \\ Director posgrado FIUBA\end{tabular} & \hspace{2cm} & \begin{tabular}[c]{@{}c@{}}\clientename \\ \empclientename \end{tabular} \vspace{2.5cm} \\ 
\multicolumn{3}{c}{\begin{tabular}[c]{@{}c@{}} \supname \\ Director del Trabajo Final\end{tabular}} \vspace{2.5cm} \\
%\begin{tabular}[c]{@{}c@{}}\jurunoname \\ Jurado del Trabajo Final\end{tabular}     &  & \begin{tabular}[c]{@{}c@{}}\jurdosname\\ Jurado del Trabajo Final\end{tabular}  \vspace{2.5cm}  \\
%\multicolumn{3}{c}{\begin{tabular}[c]{@{}c@{}} \jurtresname\\ Jurado del Trabajo Final\end{tabular}} \vspace{.5cm}                                                                     
\end{tabular}
\end{table}




\section{1. Descripción técnica-conceptual del proyecto a realizar}
\label{sec:descripcion}

\begin{consigna}{black} % El bloque "consigna" se usa para poner texto en rojo y dar una pequeña ayuda sobre cómo completar la sección. En cada entrega parcial deben eliminar los comandos begin y end del bloque consigna de las secciones que hayan completado.
El presente proyecto es un emprendimiento personal que busca volcar los conocimientos aprendidos de diseño y programación de sistemas embebidos tomando como caso de uso un robot de exploración ambiental. 

En una primera versión el dispositivo tendrá las funciones básicas de poder desplazarse, sensar el medio ambiente, ser controlado por un mando a distancia de manera cableada y comunicar las diferentes mediciones al control de mandos para su visualización.


\textit{\textbf{Estado del arte:}}
los robots exploradores son dispositivos robotizados que han sido creados con el fin de reconocer y explorar un lugar o terreno siendo capaces de moverse de forma autónoma o controlados por personas a control remoto. Su objetivo es evitar poner en riesgo la vida de los humanos, ya sea debido a que el lugar es inaccesible o porque se encuentra en una zona contaminada.
Tienen como finalidad hacer reconocimiento allí en donde el hombre no puede llegar por ser una zona inaccesible o porque supondría un peligro para la salud. También son utilizados en lugares de difícil acceso, a donde sí que podría llegar una persona solo que empleando más tiempo y recursos económicos.
Una de sus principales características es que están diseñados para moverse por terrenos con alta dificultad para desplazarse. En función de las necesidades del entorno en el que van a trabajar, disponen de diferentes sistemas de motricidad, como son los bípedos o cuadrúpedos, a los que hay sumar los que se mueven por medio de una oruga.
En cuanto a la forma de control, se pueden manejar por control remoto, habiendo equipos más sofisticados que gracias a aplicaciones controladas por Inteligencia Artificial están preparados para desplazarse y tomar decisiones de forma autónoma.
Algunos de los tipos de robots exploradores más conocidos son: robots exploradores espaciales, robots exploradores de minas, robots exploradores de rescate en catástrofes, robots exploradores de tuberías, robots exploradores acuáticos y/o submarinos, etc.

En el siguiente diagrama se puede apreciar el diseño a alto nivel del sistema embebido del robot.

\begin{figure}[htpb]
\centering 
\includegraphics[width=.9\textwidth]{./Figuras/ProyectoFinal-Page-7.jpg}
\caption{Diagrama en bloques del sistema.}
\label{fig:diagBloques}
\end{figure}

\vspace{25px}


\end{consigna}

\section{2. Identificación y análisis de los interesados}
\label{sec:interesados}
\begin{consigna}{black} % este comando se debe borrar para la entrega, junto con la contraparte \end{consigna}{red} 
A continuación se enumeran los diferentes roles e individuos que participarán en el proyecto.
\begin{table}[ht]
%\caption{Identificación de los interesados}
%\label{tab:interesados}
\begin{tabularx}{\linewidth}{@{}|l|X|X|l|@{}}
\hline
\rowcolor[HTML]{C0C0C0} 
Rol           & Nombre y Apellido & Organización 	& Puesto 	\\ \hline

Cliente       & \clientename      &\empclientename	&  -      	\\ \hline
Responsable   & \authorname       & FIUBA        	& Alumno 	\\ \hline
Orientador    & \supname	      & \pertesupname 	& Director Trabajo final \\ \hline
\end{tabularx}
\end{table}


\end{consigna} % este comando se debe borrar para la entrega, junto con la contraparte \begin{consigna}{red}



\section{3. Propósito del proyecto}
\label{sec:proposito}

\begin{consigna}{black}
El propósito de este proyecto es volcar en un caso de la industria los conocimientos más importantes aprendidos en la especialización de sistemas embebidos.

Finalmente, cabe destacar, que si bien el robot de exploración ambiental del presente proyecto es una implementación abstracta con a funcionalidades genéricas (detalladas más adelante en la sección Alcance del Proyecto), su arquitectura podría ser extrapolada a casos de uso más interesantes y de valor en la industria como por ejemplo la exploración de suelos en el agro, la exploración submarina para la perforación de pozos de petróleo, o los antes mencionados en el estado del arte.

\end{consigna}

\section{4. Alcance del proyecto}
\label{sec:alcance}

\begin{consigna}{black}
Las funcionalidades incluidas en el alcance del proyecto serán:
\begin{itemize}
	\item Sistema de desplazamiento terrestre.
	\item Operaciones de exploración (como por ejemplo medición de humedad, temperatura, presión ambiental, etc).
	\item Visualización de estado de exploración (lecturas de los sensores).
	\item Sistema de control por medio de un Joystick cableado.
\end{itemize}

Queda fuera del alcance:
\begin{itemize}
	\item locomoción por cualquier otro medio que no sea terrestre,
	\item cualquier otra función no contemplada en este alcance.
\end{itemize}
\end{consigna}


\section{5. Supuestos del proyecto}
\label{sec:supuestos}

\begin{consigna}{black}
Para el desarrollo del presente proyecto se supone que: 

\begin{itemize}
	\item será posible conseguir los componentes materiales necesarios,
	\item se dispondrá del conjunto de librerías, drivers y APIs de bajo nivel para el desarrollo de las funcionalidades planteadas en el alcance sin ser necesario el desarrollo de drivers y dichos componentes de bajo nivel,
	\item los componentes open source de la comunidad de software libre utilizados a bajo nivel para el acceso al hardware de sensores y actuadores se encontrará estable para que su integración en el proyecto no resulte en desvíos,	
	\item tanto el prototipado de los componentes de software del sistema embebido como el ensamblado de los componentes de hardware del dispositivo no producirán desvíos considerables en el plan,
	\item no habrá desvíos no contemplados en el plan que impidan o demoren entregas en el proyecto,
	\item el comité académico encargado de la corrección tendrá disponibilidad para realizar la evaluación en las fechas planificadas de entrega,
	\item el director asignado tendrá la disponibilidad de tiempo para darle seguimiento al proyecto.
	\item el alumno contará con una disponibilidad de 5 horas diarias (incluyendo fines de semana) para el desarrollo del proyecto en el tiempo convenido.
	\item los materiales y componentes adquiridos funcionaran de forma óptima y de acuerdo a lo esperado
	\item los recursos no directamente relacionados con el desarrollo del proyecto, pero utilizados durante el mismo, funcionaran adecuadamente y en caso de falta de suministro (por ejemplo el servicio de internet) o averia (por ejemplo en el caso de la computadora utilizada) la resolucion sera expeditiva no suponiendo un desvio en el plan.
	\item no sucederan nuevos eventos de impacto global (pandemia, guerra mundiale, invasión alienígena, etc) durante el desarrollo del proyecto que impliquen una demora o imposibilidad en la entrega.
\end{itemize}


\end{consigna}

\section{6. Requerimientos}
\label{sec:requerimientos}
\begin{consigna}{black}
A continuación se listan los requerimientos del producto:
\begin{enumerate}	
	\item Requerimientos funcionales		
	\begin{enumerate}			
		\item El sistema debe contar con funciones de desplazamiento para poder moverse hacia adelante y atrás, y poder girar radialmente hasta un ángulo de 360 grados.			
		\item El sistema debe ser capaz de realizar las siguientes operaciones de exploración:			
			\begin{enumerate}				
				\item medición de humedad en el ambiente,				
				\item medición de temperatura en el ambiente,				
				\item medición de luminosidad en el ambiente,				
				\item medición de presión ambiental.			
			\end{enumerate}			
		\item El sistema debe poder ser controlado a distancia mediante un joystick para que el dispositivo pueda realizar sus movimientos. En caso de que alguna de sus operaciones de exploración requiera algún mecanismo de control, el mismo también será integrado en el joystick.		
		\item El sistema debe proveer un mecanismo de visualización de las operaciones de exploración al usuario que controla el dispositivo para poder ver el estado y lectura de las operaciones de exploración.		
		\end{enumerate}	
	\item Requerimientos de documentación		
		\begin{enumerate}			
			\item Documentación de arquitectura técnica a alto nivel del diseño del sistema.			
			\item Documentación técnica de la implementación del software.			
			\item Documentación técnica de la implementación del hardware.			
			\item Manual de usuario.	
			\item Informe de avance.
			\item Memoria final.	
		\end{enumerate}	
	\item Requerimiento de testing		
		\begin{enumerate}			
			\item Se debe incluir tests de integración de componentes.		
		\end{enumerate}	
	\item Requerimientos de la interfaz		
		\begin{enumerate}			
			\item La interfaz de usuario debe permitir visualizar las lecturas de cada uno de los sensores.			
			\item Debe haber una pequeña leyenda de la magnitud que se está midiendo y la unidad utilizada junto con el valor.		
		\end{enumerate}	
	\item Requerimientos opcionales		
		\begin{enumerate}			
			\item De interfaz: se permite agregar cualquier otra interfaz adicional que agregue mejoras en la experiencia de usuario			
			\item De operaciones de exploración: se permite agregar cualquier otra operación adicional de exploración que agregue valor a exploración.	
			\item De comunicación: se permite agregar comunicación inalámbrica.		
	\end{enumerate}
\end{enumerate}




\end{consigna}

\section{7. Historias de usuarios (\textit{Product backlog})}
\label{sec:backlog}

\begin{consigna}{black}
A continuacion se listan las historias de usuario. La ponderacion de \textit{story points} se realiza considerando 1 punto = 1 hora:
\begin{itemize}
	\item (150 puntos) Como explorador quiero ver las lecturas de exploración en un display, indicando las magnitudes y unidades usadas para saber lo que el robot está midiendo.
	\item (100 puntos) Como explorador quiero que el robot pueda medir la presión ambiental para que sea de utilidad en la aplicaciones de mineria y excavación.
	\item (100 puntos) Como explorador quiero que el robot pueda medir temperatura y humedad para que sea de utilidad en aplicaciones donde sea necesario determinar dichos parámetros ambientales y una persona no pueda/deba acceder.
	\item (100 puntos) Como explorador quiero que el robot pueda medir luminosidad ambiental para poder usarlo en aplicaciones de mineria.
	\item (150 puntos) Como explorador quiero un joystick para poder controlar los movimientos del robot.
\end{itemize}
\end{consigna}

\section{8. Entregables principales del proyecto}
\label{sec:entregables}
\begin{consigna}{black}
Los entregables del proyecto son:
\begin{itemize}
	\item Documentación:
	\begin{enumerate}				
		\item Manual de usuario,			
		\item Memoria final,
		\item Informe de avance,
		\item Documentación de arquitectura técnica del sistema,
		\item Documentación técnica de diseño de software,
		\item Documentación técnica de diseño de hardware.						
	\end{enumerate}	
	\item Código fuente del firmware.
	\item Video demostrativo de uso. 
	\item Informe final.
\end{itemize}
\end{consigna}

\section{9. Desglose del trabajo en tareas}
\label{sec:wbs}

\begin{consigna}{black}
El conjunto de actividades y tareas que se realizarán durante el proyecto son:

\begin{enumerate}
\item POC (prueba de concepto), experimentación y prototipado (168 h)
	\begin{enumerate}
	\item POC de plataforma base (24 h).
	\item POC e integración de sensor de temperatura y humedad en plataforma base (24 hs).
	\item POC e integración de sensor de luminosidad en plataforma base (24 h).
	\item POC e integración de sensor de presión ambiental en plataforma base (24 h).
	\item POC e integración de motores en plataforma base (24 h).
	\item POC e integración de joystick en plataforma base (24 h).
	\item POC e integración de display en plataforma base (24 h).
	\end{enumerate}
\item Set-up ambiente de integración continua (actividad opcional) (72 h)
	\begin{enumerate}
	\item Set-up imagen Docker con código de proyecto (24 h).
	\item Set-up servicio de integración continua cloud (24 h).
	\item Configuracion con Github para tomar los commits y ejecutar los builds (24 h).
	\end{enumerate}
\item Diseño y desarrollo de funcionalidades (120 h)
	\begin{enumerate}
	\item Diseño del framework de orquestascion de componentes (24 h).
	\item Desarrollo de funcionalidad de medición de temperatura y humedad (16 h).
	\item Desarrollo de funcionalidad de medición de presión ambiental (16 h).
	\item Desarrollo de funcionalidad de medición de luminosidad (16 h).
	\item Desarrollo de funcionalidad de medición de desplazamiento (16 h).
	\item Desarrollo de funcionalidad de medición de comandos en el joystick analógico (16 h).
	\item Desarrollo de funcionalidad de escritura y formato de valores en el display (16 h).
	\end{enumerate}
\item Testing (96 h)
	\begin{enumerate}
	\item Desarrollo de tests de integración de sensores (24 h).
	\item Desarrollo de tests de integración de display (24 h).
	\item Desarrollo de tests de integración de joystick (24 h).
	\item Desarrollo de tests de integración de motores (24 h).
	\item Desarrollo de tests de integracion de producto (16 h)
	\item Tests funcional de interfaz (8 h)
	\item Tests funcional del producto final (16 h)
	\end{enumerate}
\item Ensamblado del hardware (72 h)
	\begin{enumerate}
	\item Ensamblado del joystick (36 h).
	\item Enamblado del robot (36 h).
	\end{enumerate}
\item Funcionalidades extras (actividad opcional) (72 h)
	\begin{enumerate}
	\item POC comunicación inalámbrica (36 h).
	\item POC dispositivo de video cámara integrada (36 h).
	\end{enumerate}
\item Documentación (160 h)
	\begin{enumerate}				
	\item Escritura de manual de usuario (24 h).			
	\item Escritura de memoria final (40 h).
	\item Escritura de informe de avance (12 h).
	\item Escritura de documentación de arquitectura técnica del sistema (24 h).
	\item Escritura de documentación técnica de diseño de software (24 h).
	\item Escritura de documentación técnica de diseño de hardware (24 h).		
	\item Creacion del video demostrativo de uso (12 h).				
	\end{enumerate}	
\end{enumerate}

Cantidad total de horas: (800 h)

\end{consigna}

\section{10. Diagrama de Activity On Node}
\label{sec:AoN}

\begin{consigna}{black}
A continuacion se detalla la lista de actividades que se realizaran durante el proyecto. Los tiempos están expresados en días, y como se considera una dedicación de 5 horas diarias (incluyendo fines de semana). 

\begin{table}[ht]
%\caption{Identificación de los interesados}
%\label{tab:interesados}
\begin{tabularx}{\linewidth}{@{}|l|X|l|l|l|@{}}
\hline
\rowcolor[HTML]{C0C0C0} 
Id	& Tarea           										& Duración 				 	& Dependencia	& Predecesora 	\\ \hline

1	& POC, experimentación y prototipado					& 168 h / 29 d				& -				&  -      		\\ \hline
2	& Set-up integración continua (opcional)				& 72 h / 15 d   			& -        		&  -			\\ \hline
3	& Diseño y desarrollo de funcionalidades    			& 120 h / 24 d				& 1			 	& 1				\\ \hline
4	& Testing								    			& 96 h / 19 d				& 3				& 3 , 6			\\ \hline
5	& Ensamblado del hardware				    			& 72 h / 15 d				& -				& -				\\ \hline
6	& Funcionalidades extras (opcional)						& 72 h / 15 d				& -				& - 			\\ \hline
7	& Documentación    										& 160 h / 32 d				& -			 	& -				\\ \hline

\end{tabularx}
\end{table}


Como se puede apreciar en la figura 2, el camino crítico está formado por las tareas: [1 - 3 - 4] y la duración mínima es de 74 días.


\begin{figure}[htpb]
\centering 
\includegraphics[width=.9\textwidth]{./Figuras/ProyectoFinal-Page-8.jpg}
\caption{Diagrama de \textit{Activity on Node}.}
\label{fig:diagBloques}
\end{figure}
\end{consigna}


\section{11. Diagrama de Gantt}
\label{sec:gantt}

\begin{consigna}{black}

En la figura 3, se muestra un ejemplo de diagrama de Gantt realizado con el paquete de \textit{GanttProject}. 

\begin{figure}[htpb]
\centering 
\includegraphics[width=0.95\textwidth]{./Figuras/Captura_Robot_Plan.png}
\caption{Diagrama de Gantt del proyecto.}
\label{fig:diagGantt}
\end{figure}
\end{consigna}

\section{12. Presupuesto detallado del proyecto}
\label{sec:presupuesto}

\begin{consigna}{black}

El siguiente cuadro presenta los costos en pesos estimados para el proyecto:

\end{consigna}

\begin{table}[htpb]
\centering
\begin{tabularx}{\linewidth}{@{}|X|c|r|r|@{}}
\hline
\rowcolor[HTML]{C0C0C0} 
\multicolumn{4}{|c|}{\cellcolor[HTML]{C0C0C0}COSTOS DIRECTOS} \\ \hline
\rowcolor[HTML]{C0C0C0} 
Descripción &
  \multicolumn{1}{c|}{\cellcolor[HTML]{C0C0C0}Cantidad} &
  \multicolumn{1}{c|}{\cellcolor[HTML]{C0C0C0}Valor unitario} &
  \multicolumn{1}{c|}{\cellcolor[HTML]{C0C0C0}Valor total} \\ \hline
 ESP32 & 
  \multicolumn{1}{c|}{1} &
  \multicolumn{1}{c|}{\$5.000} &
  \multicolumn{1}{c|}{\$5.000} \\ \hline
 Joystick analógico &
  \multicolumn{1}{c|}{1} &
  \multicolumn{1}{c|}{\$415} &
  \multicolumn{1}{c|}{\$415} \\ \hline
 Rueditas &
  \multicolumn{1}{c|}{4} &
  \multicolumn{1}{c|}{\$480} &
  \multicolumn{1}{c|}{\$1.920} \\ \hline
 Sensor BMP280 &
  \multicolumn{1}{c|}{1} &
  \multicolumn{1}{c|}{\$1.255} &
  \multicolumn{1}{c|}{\$1.255} \\ \hline
 Sensor DHT22 &
  \multicolumn{1}{c|}{1} &
  \multicolumn{1}{c|}{\$2.300} &
  \multicolumn{1}{c|}{\$2.300} \\ \hline
 Fotoresitor &
  \multicolumn{1}{c|}{1} &
  \multicolumn{1}{c|}{\$1.220} &
  \multicolumn{1}{c|}{\$1.220} \\ \hline
 Motores DC 3-6 V &
  \multicolumn{1}{c|}{4} &
  \multicolumn{1}{c|}{\$990} &
  \multicolumn{1}{c|}{\$3.960} \\ \hline
 Cables dupont macho-macho &
  \multicolumn{1}{c|}{1} &
  \multicolumn{1}{c|}{\$960} &
  \multicolumn{1}{c|}{\$960} \\ \hline
 Cables dupont macho-hembra &
  \multicolumn{1}{c|}{1} &
  \multicolumn{1}{c|}{\$960} &
  \multicolumn{1}{c|}{\$960} \\ \hline
 Plaqueta de cobre para montar &
  \multicolumn{1}{c|}{1} &
  \multicolumn{1}{c|}{\$1.250} &
  \multicolumn{1}{c|}{\$1.250} \\ \hline

\multicolumn{3}{|c|}{TOTAL} &
  \multicolumn{1}{c|}{\$19.240} \\ \hline
\rowcolor[HTML]{C0C0C0} 
% \multicolumn{4}{|c|}{\cellcolor[HTML]{C0C0C0}COSTOS INDIRECTOS} \\ \hline
% \rowcolor[HTML]{C0C0C0} 
% Descripción &
%   \multicolumn{1}{c|}{\cellcolor[HTML]{C0C0C0}Cantidad} &
%   \multicolumn{1}{c|}{\cellcolor[HTML]{C0C0C0}Valor unitario} &
%   \multicolumn{1}{c|}{\cellcolor[HTML]{C0C0C0}Valor total} \\ \hline
% \multicolumn{1}{|l|}{-} &
%   - &
%   - &
%   -\\ \hline
% \multicolumn{1}{|l|}{-} &
%   - &
%   - &
%   - \\ \hline
% \multicolumn{1}{|l|}{-} &
%   - &
%   - &
%   - \\ \hline
% \multicolumn{3}{|c|}{SUBTOTAL} &
%   \multicolumn{1}{c|}{-} \\ \hline
% \rowcolor[HTML]{C0C0C0}
% \multicolumn{3}{|c|}{TOTAL} & \$19.240
%    \\ \hline
\end{tabularx}%
\end{table}




\section{13. Gestión de riesgos}
\label{sec:riesgos}

\begin{consigna}{black}

Se han identificado los siguientes riesgos:


\begin{enumerate}

\item Riesgo de demora
\begin{itemize}
	\item Severidad (S): 9 - Teniendo en cuenta que solo habrá un recurso desarrollando el proyecto, una demora en cualquier tarea puede implicar demora en la fecha de entrega.
	\item Ocurrencia (O): 3 - Dada la planificación y priorización de tareas, el riesgo de demora es bajo, pero no nulo.
\end{itemize}


\item Riesgo de no contar con toda la funcionalidad deseada
\begin{itemize}
	\item Severidad (S): 9 - El no cumplimiento con la funcionalidad deseada pone en riesgo el éxito del proyecto.
	\item Ocurrencia (O): 1 - La probabilidad de no poder cumplir con la funcionalidad deseada es realmente bajo teniendo en cuenta que tanto el alumno como el director tienen experiencia en el desarrollo de integraciones similares y los materiales necesarios son de fácil obtención.
\end{itemize}

\item Riesgo de calidad insuficiente
\begin{itemize}
	\item Severidad (S): 4 - La calidad insuficiente no pone en riesgo el cumplimiento con la funcionalidad pero si compromete la estabilidad y resistencia a fallas del producto, por lo que puede desencadenar en un producto poco o menos confiable.
	\item Ocurrencia (O): 4 - Se estima que con las técnicas empleadas durante el desarrollo del producto, este riesgo tiene una baja probabilidad de ocurrencia.
\end{itemize}


\item Riesgo de desvío en costos
\begin{itemize}
	\item Severidad (S): 5 - La realización de este riesgo impacta en los costos del proyecto, pero no imposibilitan ni demora la entrega del mismo.
	\item Ocurrencia (O): 8 - Teniendo en cuenta que los precios son estimados en base a los productos que se logran identificar al momento de realizar el presente plan, es altamente posible que haya artículos adicionales que requieran ser comprados y/o que los costos finales difieran de los esperados. Además de lo antes mencionado, existe el factor inflación argentina como otro disparador de este riesgo.
\end{itemize}

\item Riesgo de indisponibilidad de recursos
\begin{itemize}
	\item Severidad (S): 5 - Al momento de realizar el presente plan se identifican ciertos recursos y se asume que será posible disponer de ellos. No obstante, existe el riesgo de que esto no suceda así, y sea más difícil por ejemplo adquirir ciertos materiales, o haya recursos no directamente asociados al proyecto pero cuya ausencia lo afectan, como por ejemplo fallas en el acceso a internet, el mal funcionamiento de la computadora utilizada para su desarrollo, etc.
	\item Ocurrencia (O): 2 - Se espera que la probabilidad de ocurrencia de este riesgo sea realmente baja.
\end{itemize}

\end{enumerate}

b) Tabla de gestión de riesgos:      (El RPN se calcula como RPN=SxO)

\begin{table}[htpb]
\centering
\begin{tabularx}{\linewidth}{@{}|X|c|c|c|c|c|c|@{}}
\hline
\rowcolor[HTML]{C0C0C0}
Riesgo 													& S & O & RPN & S* & O* & RPN* \\ \hline
Riesgo de demora en la entrega							& 9 & 3 & 27 &  9  &  3  & 27    \\ \hline
Riesgo de no contar con toda la funcionalidad deseada	& 9 & 2 & 18  & 7  & 2 &  14    \\ \hline
Riesgo de calidad insuficiente							& 4 & 4 & 16 &  - &  - &   -  \\ \hline
Riesgo de desvío en costos								& 5 & 8 & 40 & 5  & 3  &  15   \\ \hline
Riesgo de indisponibilidad de recursos					& 5 & 2 & 10 & -  & -  &   -   \\ \hline
\end{tabularx}%
\end{table}

Criterio adoptado:
Se tomarán medidas de mitigación en los riesgos cuyos números de RPN sean mayores a 15.

Nota: los valores marcados con (*) en la tabla corresponden luego de haber aplicado la mitigación.

c) Plan de mitigación de los riesgos que originalmente excedían el RPN máximo establecido:
\begin{enumerate}


	\item Riesgo de demora en la entrega: Las posibles causas del evento asociado están vinculadas a situaciones no controlables ni predecibles que impactan de alguna manera en la disponibilidad de tiempo o alguno de los recursos necesarios para la realización del proyecto. Por este motivo se considera que la única herramienta de mitigación disponible es o pedir una prórroga para la entrega, o aceptar el riesgo.
	\begin{itemize}
		\item Nueva Severidad (S*): 9 - No cambia.
		\item Nueva Ocurrencia (O*): 3 - No cambia.
	\end{itemize}
	
	\item Riesgo de no contar con toda la funcionalidad deseada: Las posibles causas del evento asociado están vinculadas a situaciones que impactan de alguna manera en la viabilidad o desarrollo de alguna de las funcionalidades en el tiempo planificado Por este motivo se considera como herramienta de mitigación sacrificar algún otro entregable, como por ejemplo la cobertura de testing (lo cual sacrifica calidad), o la documentación.
	\begin{itemize}
		\item Nueva Severidad (S*): 7 - Dado que tras la mitigación se incrementa el impacto por pérdida de calidad.
		\item Nueva Ocurrencia (O*): 2 - Se reduce mucho la probabilidad de ocurrencia dado que se agrega tiempo para el desarrollo de funcionalidad eliminando el tiempo empeñado para el desarrollo de tests o documentación.
	\end{itemize}
	
	
	\item Riesgo de calidad insuficiente: Las posibles causas del evento asociado están vinculadas a la falta de estabilidad del producto. Para mitigar este problema se plantea incrementar las prácticas de testing y CI/CD (como actividad opcional) siempre que esto no dispare el riesgo 1 generando una demora en el proyecto.
	\begin{itemize}
		\item Nueva Severidad (S*): 4 - No cambia.
		\item Nueva Ocurrencia (O*): 2 - Se reduce la probabilidad de que esto suceda.
	\end{itemize}	
	
	\item Riesgo de desvío en costos: Las posibles causas del evento que dispara este riesgo son olvidar de estimar algún componente y la inflación. Para mitigar el segundo factor se plantean los precios en dólares americanos, dado que la inflación de los precios en dicha divisa sera inferior a los representados en pesos argentinos.
	\begin{itemize}
		\item Nueva Severidad (S*): 5 - Esto no varía.
		\item Nueva Ocurrencia (O*): 3 - Se reduce mucho la probabilidad de ocurrencia dado que la inflación del USD es menor que la del peso.
	\end{itemize}

	
\end{enumerate}

\end{enumerate}
\end{consigna}


\section{14. Gestión de la calidad}
\label{sec:calidad}
\begin{consigna}{black}
A continuación se detalla cómo se realizará el control de calidad para cada uno de los requerimientos del producto:
\begin{enumerate}	
			
		\item Funciones de desplazamiento (hacia adelante, atrás, y radialmente en 360 grados)
		\begin{enumerate}				
			\item Verificación previo a la entrega: se verificará mediante la ejecución de tests de integración para esta funcionalidad.			
			\item el cliente validará la funcionalidad en el producto final.			
		\end{enumerate}		
		\item Operaciones de exploración ( medición de humedad, temperatura, luminosidad y presión ambiental)
		\begin{enumerate}				
			\item Verificación previo a la entrega: se verificará mediante la ejecución de tests de integración para esta funcionalidad.			
			\item el cliente verifica la funcionalidad en el producto final.			
		\end{enumerate}		
		\item Control a distancia mediante joystick
		\begin{enumerate}				
			\item Verificación previo a la entrega: se verificará mediante la ejecución de tests de integración para esta funcionalidad.			
			\item el cliente verifica la funcionalidad en el producto final.			
		\end{enumerate}			
		\item Visualización/reporte de las operaciones de exploración 
		\begin{enumerate}				
			\item Verificación previo a la entrega: se verificará mediante la ejecución de tests de integración para esta funcionalidad.			
			\item el cliente verifica la funcionalidad en el producto final.			
		\end{enumerate}			
		\item Documentación técnica, manual de usuario, informe de avance y memoria final
		\begin{enumerate}				
			\item Verificación previo a la entrega: se verificará mediante la revisión de los documentos.			
			\item el cliente verificará los documentos.			
		\end{enumerate}			
		\item Testing
		\begin{enumerate}				
			\item Verificación previo a la entrega: Se verificará el cumplimiento con los tests por funcionalidad previo la integración de cada componente en el prototipo final. 
			\item El cliente validará el reporte de los tests de integración.
		\end{enumerate}			
		\item Requerimientos de la interfaz		
		\begin{enumerate}			
			\item Verificación previo a la entrega: Se verificará el cumplimiento con los requerimientos de interfaz mediante un smoke test, además del test funcional final sobre el prototipo integrado.			
			\item El cliente verificará el cumplimiento con la interfaz sobre el prototipo final.		
		\end{enumerate}	
		\item Para los requerimientos adicionales (de interfaz, operaciones y comunicación)
		\begin{enumerate}			
			\item Una vez realizada una prueba de concepto de su viabilidad, serán prototipados y se realizará el correspondiente test de integración. El cliente confirmará si la funcionalidad provista en la prueba de concepto se ajusta al requerimiento opcional. Si esto es así, se integrará al prototipo final y se verificará el funcionamiento.
			\item El cliente verifica la funcionalidad como parte del prototipo final
	\end{enumerate}
\end{enumerate}


\end{consigna}

\section{15. Procesos de cierre}   
\label{sec:cierre}

\begin{consigna}{black}
A continuación se detallan las pautas y actividades para realizar la reunión final de evaluación del proyecto:

\begin{itemize}
	\item Pautas de trabajo que se seguirán para analizar si se respetó el Plan de Proyecto original:
	 - Responsable: \authorname:
	\begin{itemize}			
		\item Se evaluaran los requerimientos y los objetivos alcanzados frente a los planteados en el plan
		\item Se pondrá especial interés en verificar si se cumplieron los objetivos de tiempo y funcionalidad propuestos.
	\end{itemize}	    	 
	
	\item Identificación de las técnicas y procedimientos útiles e inútiles que se emplearon, y los problemas que surgieron y cómo se solucionaron:
	 - Responsable: \authorname:
	\begin{itemize}			
		\item Se evaluará cuál fue la configuración que mejores resultados arroj́a, para los objetivos planteados en el plan.
		\item Se identificar ́an nuevas herramientas o procedimientos, en caso que corresponda.
	\end{itemize}	    	 
	
	\item Indicar quién organizará el acto de agradecimiento a todos los interesados, y en especial al equipo de trabajo y colaboradores - Responsable: \authorname :
	\begin{itemize}			
		\item Luego de la presentación del proyecto mediante la defensa pública , se procederá a agradecer a todas las personas que participaron del desarrollo del proyecto, al director, a los compañeros y a las autoridades del CESE.
	\end{itemize}	 
	
	    	 
	 
\end{itemize}

\end{consigna}




\end{document}
